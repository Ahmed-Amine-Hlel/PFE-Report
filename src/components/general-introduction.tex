\thispagestyle{plain} % Remove header
\addcontentsline{toc}{chapter}{General Introduction}
\section*{General Introduction}
In the dynamic field of Information Technology (IT), the recruitment and development of professional skills pose significant challenges. The i Competency Dictionary (iCD), a toolset developed by the Japan IT Standards Organization, offers a comprehensive framework for understanding and developing IT skills. However, its potential remains largely untapped due to its complexity and lack of user-friendly interfaces.

The iCD toolset is a robust and detailed system, encompassing a wide range of IT skills and tasks. It provides a structured approach to identifying and categorizing IT skills, making it a valuable resource for HR managers and IT professionals. However, its complexity and the lack of a user-friendly interface have limited its adoption and use.

This report presents a project titled "Japan IT Skills System Simplified", which aims to address these challenges. The project's goal is to simplify the iCD toolset and develop a user-friendly web or mobile application. This application is designed to serve internal and external IT recruiters, as well as IT professionals, enabling them to assess IT skill needs effectively.

The project involves translating the complex iCD data into a more accessible format, automating the process of fetching relevant skills and tasks based on the selected job category, and leveraging various technologies and methodologies to enhance the usability of the iCD toolset. The project also aims to integrate the application with existing HR systems and job posting platforms, making it a seamless part of the IT recruitment and skill development process.

The report is structured as follows: After this introduction, the report will delve into the current state of the iCD toolset, followed by a detailed description of the project and its objectives. Subsequent sections will discuss the methodologies used, the technologies leveraged, and the resources involved in the project. The report will conclude with a summary of the project's outcomes, user feedback, and recommendations for further improvements.

\newpage

