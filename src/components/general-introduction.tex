\thispagestyle{empty}
\addcontentsline{toc}{chapter}{General Introduction}
\section*{General Introduction}
Companies are always looking for ways to grow and it is the more so especially in the IT world where technology evolves at an insane rate. The main way of achieving growth for any company is to gain the trust of their customers whose satisfaction is at the center of its concerns.
And what's better to gain customers' trust than improving the software quality, added value as well as delivery times (Time To Market) or in other words, adhere to the DevOps principles.

It is in light of this background that our End of Studies Project was designed. The goal of the project is to build on the already existing proprietary SaaS (Software as a Service) developed by the company Incedo Services GmbH; it is about further developing the software by delivering new features, fixing bugs and migrating the whole application infrastructure.
Since the previous deployment is bottlenecked due to the increase in customers, we have to make the software in question infinitely scalable by separating it to multiple microservices while ensuring the service to service communication.
It is imperative on top of that that we setup a fully automated DevOps workflow for continuously monitoring the state of the services and where any changes made to the codebase are delivered to customers on the fly.
For that, we have to make the deployment on cloud servers that we additionally need to setup, configure and manage separately.

This report contains four chapters, the first of which is dedicated to introducing our project where we present the general framework of the internship, the host company as well as the assessment of the work.
The second chapter is devoted to defining the basic concepts in our project in addition to making comparative studies between the various tools where we explain some of our technical choices where relevant.
The third chapter will pick up the pace by setting out the analysis and specifications of the functional and non-functional requirements in order to specify the objectives we want to achieve. The forth and final chapter will then present the implementation phase where we go a bit more into the technical details of the project such as the hardware and software setup in addition to the deployment process in addition to the incorporation of DevOps practices.

\newpage
