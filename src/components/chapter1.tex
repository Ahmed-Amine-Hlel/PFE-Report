\chapter{Context of the work}
\setcounter{secnumdepth}{3}
\newpage

% ----------------------- Introduction ----------------------- %
\section*{Introduction}
This chapter introduces the general context of this report. We start by presenting the frame of the project as well as the host company. Then comes the enumeration of the problems which led to the realization of the project. We wrap it up by defining the methodology we’ve followed to carry out our work. \citep{test1}

% ----------------------- General framework of the internship ----------------------- %
\section{General framework of the internship}
This project was carried out within the frame of obtaining a bachelor’s degree in Computer Science at the Higher Institute of Technological studies of Nabeul. The internship took place fully remotely at Incedo Services GmbH for five months starting from the 1st of February 2022 to the 30th of June 2022 with the purpose of further developing the existing project as well as slowly migrating it to a micro-services architecture. \cite{test2}

% ----------------------- Company overview ----------------------- %
\section{Company overview}
This section introduces the host company and its offered services.
\subsection{About Incedo}
\subsection{Incedo Services}

% ----------------------- Stating the problem ----------------------- %
\section{Stating the problem}
Incedo -having ambitious goals to grow over the next 2 to 4 years- wants to win new clients and strategically develop the existing ones. Winning new clients starts with generating new leads. Previously, Incedo has worked with an Austrian start-up (motion group) that provides automated lead generation through LinkedIn at the cost of 0.85 € per requested contact in LinkedIn. Although one big project was closed and several leads were generated, Incedo started using the LinkedIn Sales Navigator with a more targeted approach, but nevertheless wanted to automate the approach and increase its outreach.
Having launched the first version of the ILG (Incedo Lead Generator) as a SaaS (Software as Service), Incedo was satisfied for a while. Over the time however, as more and more clients were interested in the ILG, the current architecture couldn’t handle the load properly.

% ----------------------- Assessment of the case ----------------------- %
\section{Assessment of the case}
\subsection{Describing the work procedure}
The work on any project must first of all be preceded by a thorough study of the existing ones which undermines the strengths and weaknesses of the current system, as well as the business decisions that should be taken into account during the conception as well as the realization.
\subsection{Criticizing the current state}
After studying the existing, we can determine its limitations:
\begin{itemize}
	\item Bugs always tend to happen whenever the LinkedIn website changes (due to scraping).
	\item Since cron jobs are running for the whole day, bugs are hard to respond to fast enough because we can only deploy once at the end of day.
	\item It is hard to test the whole workflow because of how the app works (looking for certain changes in the LinkedIn interface after certain buttons are clicked for example) meaning that the dev environment is lacking.
	\item It cannot scale well enough since the whole application is deployed on a single server.
\end{itemize}
\subsection{Proposed solution}
% NOTE: Fix New Paragraph identation
The solution to these problems is refactoring the whole application to separate sub applications (microservices) where the automation and scraping processes can be scaled independently of the other parts of the application.

\medskip
This way, it will be easier to maintain and scale the different codebases as well as respond faster to bugs and know exactly what caused them in the first place.

\newpage

% ----------------------- Development Methodology ----------------------- %
\section{Development Methodology}
\subsection{Agile methodology}
Agile is a structured and iterative approach to project management and product development. It recognizes the volatility of product development, and provides a methodology for self-organizing teams to respond to change without going off the rails.
\subsection{Scrum methodology}
Scrum teams commit to completing an increment of work, which is potentially shippable, through set intervals called sprints. Their goal is to create learning loops to quickly gather and integrate customer feedback. Scrum teams adopt specific roles, create special artifacts, and hold regular ceremonies to keep things moving forward.
\subsection{Kanban methodology}
Kanban is all about visualizing your work, limiting work in progress, and maximizing efficiency (or flow). Kanban teams focus on reducing the time a project takes (or user story) from start to finish. They do this by using a kanban board and continuously improving their flow of work.
\subsection{The choice for ILG}
Kanban is based on a continuous workflow structure that keeps teams nimble and ready to adapt to changing priorities. Work items—represented by cards— are organized on a kanban board where they flow from one stage of the workflow(column) to the next. Common workflow stages are To Do, In Progress, In Review, Blocked, and Done. And since this project is very susceptible to changes from outside (LinkedIn), Kanban offered more flexibility than Scrum so that’s why the team went with it instead.
