\chapter{Analysis and specification of requirements:}
\newpage

\setcounter{secnumdepth}{0} % Set the section counter to 0 so next section is not counted in toc
% ----------------------- Introduction ----------------------- %
\section{Introduction}
This chapter will present and study various concepts such as Automation, Web Scraping, Version Control, DevOps, State Machine with an emphasis on the DevOps terminology such as the CI/CD pipelines. We will talk about the most used tools in the market as well as which ones we settled on after making a comparison.

\setcounter{secnumdepth}{2} % Resume counting the sections for the toc with a depth of 2 (Sections and sub-sections)
% ----------------------- Analysis ----------------------- %
\section{Analysis}

% ----------------------- Requirements ----------------------- %
\section{Requirements}
\subsection{SprintOne}
\subsubsection{Sprint Story and Objective}
\subsubsection{Sprint Analysis}
\subsubsection{Sprint Kanban Board}
\subsubsection{Sprint Review}

% TODO : CHANGE_ME
\subsection{SprintTwo}
\subsubsection{Sprint Story and Objective}
\subsubsection{Sprint Analysis}
\subsubsection{Sprint Kanban Board}
\subsubsection{Sprint Review}

% TODO : CHANGE_ME
\subsection{SprintThree}
\subsubsection{Sprint Story and Objective}
\subsubsection{Sprint Analysis}
\subsubsection{Sprint Kanban Board}
\subsubsection{Sprint Review}

\setcounter{secnumdepth}{0} % Set the section counter to 0 so next section is not counted in toc
% ----------------------- Conclusion ----------------------- %
\section{Conclusion}
This chapter introduces the general context of this report. We start by presenting the frame of the project as well as the host company. Then comes the enumeration of the problems which led to the realization of the project. We wrap it up by defining the methodology we’ve followed to carry out our work.
